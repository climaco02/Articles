\documentclass[12pt]{article}

\usepackage{sbc-template}

\usepackage{graphicx,url}

\usepackage[brazil]{babel}   
\usepackage[latin1]{inputenc}  

     
\sloppy

\title{Jogos de Apoio ao Ensino da disciplina de IHC}

\author{Andr� Barros de Sales\inst{1}, Gabriel de Souza Cl�maco\inst{1}}

\address{
Universidade de Bras�lia - Faculdade do Gama
  \email{andrebds@unb.br, gabrielsclimaco@gmail.com}
}

\begin{document} 

\maketitle

\begin{abstract}

\end{abstract}
     
\begin{resumo} 

\end{resumo}


\section{Introdu��o}

\section{Metodologia Utilizada}

A metodologia utilizada durante todo o projeto de inicia��o cient�fica foi
dividida em tr�s partes. A primeira parte tratou da busca pelos artigos j� 
existentes dentro da �rea de IHC; j� a segunda parte diz respeito a compara��o 
entre os artigos encontrados; por fim, replicou-se o artigo que mais se 
adequava � realidade da universidade.

Na primeira parte foi realizada uma revis�o sistem�tica, uma metodologia de 
estudo secund�rio estabelecer um levantamento formal do estado da arte de forma
robusta e consistente, a partir de um planejamento e execu��o criteriosos 
\cite{biol:05}. Objetivou-se, na revis�o em quest�o, encontrar o maior n�mero 
de artigos poss�veis e que sejam pass�veis de aplica��o na disciplina de IHC. 
A revis�o sistem�tica foi subdividida em tr�s etapas: planejamento, execu��o 
da pesquisa e an�lise dos resultados obtidos \cite{sales:16}.

Na segunda parte da metodologia 

\section{Resultados e Discuss�es}

\section{Considera��es finais}

\bibliographystyle{sbc}
\bibliography{sbc-template}

\end{document}
